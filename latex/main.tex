\documentclass{article}


% if you need to pass options to natbib, use, e.g.:
%     \PassOptionsToPackage{numbers, compress}{natbib}
% before loading neurips_2023


% ready for submission
\usepackage[final]{neurips_2023}


% to compile a preprint version, e.g., for submission to arXiv, add add the
% [preprint] option:
%     \usepackage[preprint]{neurips_2023}


% to compile a camera-ready version, add the [final] option, e.g.:
%     \usepackage[final]{neurips_2023}


% to avoid loading the natbib package, add option nonatbib:
%    \usepackage[nonatbib]{neurips_2023}


\usepackage[utf8]{inputenc} % allow utf-8 input
\usepackage[T1]{fontenc}    % use 8-bit T1 fonts
\usepackage{hyperref}       % hyperlinks
\usepackage{url}            % simple URL typesetting
\usepackage{booktabs}       % professional-quality tables
\usepackage{amsfonts}       % blackboard math symbols
\usepackage{nicefrac}       % compact symbols for 1/2, etc.
\usepackage{microtype}      % microtypography
\usepackage{xcolor}         % colors
\usepackage{graphicx}       % images
\usepackage{float}          % positioning
\graphicspath{ {./img/} }


\title{Proyecto Final Aprendizaje Automático}


% The \author macro works with any number of authors. There are two commands
% used to separate the names and addresses of multiple authors: \And and \AND.
%
% Using \And between authors leaves it to LaTeX to determine where to break the
% lines. Using \AND forces a line break at that point. So, if LaTeX puts 3 of 4
% authors names on the first line, and the last on the second line, try using
% \AND instead of \And before the third author name.


\author{%
    Lucía Herraiz Cano\\
    Aprendizaje Automático\\
    Universidad Pontificia Comillas\\
    Abril 2025\\
    \texttt{202300465@alu.comillas.edu} \\
}


\begin{document}


\maketitle


\begin{abstract}
  En este proyecto se desarrolla un análisis predictivo sobre el rendimiento 
  académico de estudiantes de secundaria en dos institutos de Madrid durante 
  el año 2005. A partir de un conjunto de datos, se construyen y comparan dos 
  modelos para predecir la nota final del curso (T3). También se lleva a cabo 
  un estudio de los datos y de las variables más influyentes en el desarrollo
  del estudiante. Este estudio busca no solo obtener predicciones precisas, 
  sino también generar conocimiento accionable para mejorar el rendimiento 
  estudiantil desde una perspectiva integral.
\end{abstract}


\section{Exploratory data analysis}


En esta sección se analiza la estructura y distribución del conjunto de datos, identificando patrones, valores atípicos y relaciones relevantes entre variables. También se detalla el proceso de limpieza y preparación necesario para el modelado posterior.

El desarrollo detallado de estos análisis se encuentra principalmente documentado en \textit{Exploratory\_Data\_Analysis}. No obstante, ciertas observaciones derivadas del análisis 
conjunto con modelos predictivos pueden consultarse en \textit{Model1\_Testing} y \textit{Model2\_Testing}.


\subsection{Limpieza de datos}


El proceso de limpieza de datos viene recogido en la función \texttt{data\_cleaning\_pipeline}
e incluye el manejo de outliers, la imputación de valores faltantes, la corrección de errores, la codificación de las variables 
categóricas y la estandarización de los datos.

Sólo 5 variables tenían valores faltantes, y con el objetivo de perder la menor cantidad de datos, todos los valores se imputaron siguiendo distintas estrategias.
\textit{AlcSem, Relfam} y \textit{TiempoEstudio} se imputaron por la moda al ser variables categóricas, y tener menos de un 3\% de valores faltantes. \textit{Medu} y \textit{Pedu}, 
al tener un porcentaje más relevante de valores faltantes, tener una correlación de Pearson alta (0.653), y al ser variables con mucho peso, como se verá posteriormente, se imputaron con un regresor (IterativeImputer) que emplea el resto de datos
para predecir sus valores de manera más robusta. 

\begin{figure}[ht]
  \centering
  \includegraphics[width=0.7\textwidth]{Boxplot_Outliers.png}
  \caption{Boxplot de las variables}\label{fig:boxplot}
\end{figure}



Respecto a los outliers, la variable más llamativa fue \textit{faltas}. Se estableció un valor máximo de 150, correspondiente al número de días lectivos del calendario escolar estándar, considerando 
cualquier valor superior como erróneo. Se probaron dos estrategias adicionales, eliminar registros, e imputar los outliers por la mediana, pero ambas estrategias disminuyeron la 
\textit{performance} de los modelos en varios puntos. Esto muestra no sólo que \textit{faltas} es una variable altamente relevante, sino que en este caso los outliers son muy informativos.

Se corrigieron los datos de la columna \textit{razon} al tener claves distintas para el mismo valor ("otras", "otros").

Finalmente, tras estudiar el balance de las clases, se vió que \textit{EstPadres, EstSup} y \textit{apoyo} estaban claramente desbalanceadas (80\%-20\%). Sin embargo, dado que ninguna es una variable
muy relevante, y dado que los resultados reflejan bien el balance que se suele dar en la población real, se decidió mantener los datos y tener cuidado con los modelos que le den relevancia a estas variables.

\subsection{Análisis no supervisado}


Con el objetivo de comprender mejor la relevancia y relación de las variables, se implementaron técnicas de aprendizaje no supervisado. 

En primer lugar, se ha implementado \textbf{Principal Component Analysis (PCA)} con el objetivo de estudiar la reducción de dimensionalidad y los componentes principales. 
La varianza explicada con 2 y 3 componentes principales es menor al 30\%, sin embargo, el estudio de sus loadings aporta información relevante. Para el 1º modelo, como cabía esperar, las variables
más relevantes son \textit{T1, T2} y \textit{suspensos}, pero es más interesante estudiar el 2º modelo, puesto que al eliminar estas variables, las sutituyen \textit{AlcSem, AlcFin, SalAm, TiempoLib} y otras, que
 indican que después de las \textbf{notas}, la \textbf{vida social del estudiante} es uno de los factores más relevantes para su desempeño. Además de estos, un tercer grupo de variables de peso que aparecen en los PC son \textit{Medu, Pedu, Relfam} y empleando \textbf{Recursive Feature Elimination (RFE)} 
se obtiene también \textit{Mtrab\_docencia} y \textit{Ptrab\_docencia}, lo que indica que el \textbf{entorno familiar} es otro de los factores más influyentes en las notas. 

\begin{figure}[h!]
  \centering
  \begin{minipage}[b]{0.45\textwidth}
      \centering
      \includegraphics[scale=0.30]{PCA2PC.png}
      \caption*{(a) PCA con 2 componentes principales}
  \end{minipage}
  \hfill
  \begin{minipage}[b]{0.45\textwidth}
      \centering
      \includegraphics[scale=0.28]{PCA3PC.png}
      \caption*{(b) PCA con 3 componentes principales}
  \end{minipage}
  \caption{Visualización de los datos proyectados sobre 2 y 3 componentes principales mediante PCA.}
\end{figure}


Sin embargo, PCA requiere 16 componentes para explicar el 80\% de la varianza. Al combinarlo con modelos como SVR o regresión lineal, su rendimiento disminuyó, por lo que se descartó su uso 
más allá de la exploración inicial. Este comportamiento podría deberse a la naturaleza lineal de PCA, incapaz de capturar relaciones no lineales en los datos. Por ello, se probaron 
técnicas no lineales como \textbf{ISOMAP} y \textbf{Kernel-PCA}, que operan en espacios transformados mediante distancias geodésicas o kernels y que pueden capturar relaciones más complejas entre los datos. No obstante, su combinación con modelos predictivos redujo el rendimiento entre un 20-30\%. 
Podemos conluir que la reducción de dimensionalidad implica la pérdida de información relevante para la predicción. Las variables no son fácilmente separables y por ello, finalmente, se empleó el dataset completo para el entrenamiento de los modelos.

\begin{figure}[h!]
  \centering
  \begin{minipage}[b]{0.45\textwidth}
      \centering
      \includegraphics[scale=0.28]{KernelPCA3.png}
      \caption*{(a) Kernel-PCA con 3 PC}
  \end{minipage}
  \hfill
  \begin{minipage}[b]{0.45\textwidth}
      \centering
      \includegraphics[scale=0.28]{ISOMAP3.png}
      \caption*{(b) ISOMAP con 3 PC}
  \end{minipage}
  \caption{Aplicación de Kernel-PCA e ISOMAP con 3 PC}
\end{figure}

Asimismo, se aplicaron técnicas de \textbf{Clustering} a los datos (combinadas con PCA). Utilizando la métrica del \textit{Elbow Method} y de la \textit{Silueta}, se obtuvo el número óptimo de clusters (2-3), que coincide con los grupos de variables relevantes identificados anteriormente. No obstante, los valores obtenidos tras aplicar K-Means (Silueta: 0.233; Índice de Davies-Bouldin: 1.341) indican una calidad media en la segmentación, reafirmando la dificultad al separar los datos. 
Experimentalmente, se probó a añadir una nueva variable indicando la pertenecia a los clusters de los datos para entrenar a un regresor lineal, pero esto no modificó su \textit{performance}, por lo que finalmente, usando la filosofía de la \textit{Navaja de Occam},
se descartó el uso activo del clustering en los modelos.

\begin{figure}[h!]
  \centering
  \begin{minipage}[b]{0.45\textwidth}
      \centering
      \includegraphics[scale=0.40]{Elbow_Method.png}
      \caption*{(a) Elbow Method}
  \end{minipage}
  \hfill
  \begin{minipage}[b]{0.45\textwidth}
      \centering
      \includegraphics[scale=0.4]{Clusters.png}
      \caption*{(b) Clusters}
  \end{minipage}
  \caption{Análisis por clustering}
\end{figure}

Aunque las técnicas de aprendizaje no supervisado no fueron incorporadas directamente en los modelos finales, su aplicación contribuyó significativamente a una comprensión más profunda de la estructura y relaciones entre las variables del conjunto de datos, lo cual resultó de gran valor para el desarrollo y justificación del enfoque predictivo adoptado.
La conclusión más relevante, utilizar el dataset entero sin reducir la dimensionalidad, llevó al descarte de modelos como KNNeighbours, que podían verse afectados por el \textit{curse of dimensionality}.

\subsection{Análisis adicional}


En paralelo al análisis anterior, y con el objetivo de ampliar el conocimiento sobre las variables, se utilizaron las capacidades explicativas de distintos modelos para evaluar la importancia relativa de cada característica.

Se estudió la representación conjunta de cada par de variables, empleando como código de color las distintas clases (0-20) para identificar posibles relaciones entre los datos\footnote{Ver la matriz completa en el fichero \textit{Exploratory\_Data\_Analysis}}, pero sólo se encontró un patrón claro con \textit{T1} y \textit{T2}, 
demostrando de nuevo que son variables muy significativas para la predicción.

\begin{figure}[ht]
  \centering
  \includegraphics[width=0.9\textwidth]{Plot_Var_T1T2.png}
  \caption{Scatter Plot con T1 y T2}\label{fig:scatter}
\end{figure}

Uno de los algoritmos con mejor desempeño en ambos modelos, como se verá más adelante, fue \textbf{Random Forest}. Vemos que las variables a las que asigna un mayor peso coinciden con las obtenidas con PCA y con otros métodos. Como T1 y T2 están muy correlacionadas (0.863)\footnote{Coeficiente de Correlación de Pearson},
vemos que el modelo asigna la mayor parte del peso a T2, lo que no indica que tenga una mayor importancia.


\begin{figure}[ht]
  \centering
  \includegraphics[width=0.3\textwidth]{RF_var.png}
  \caption{Feature importance en Random Forest}\label{fig:t1t2}
\end{figure}

A pesar de que el modelo de \textbf{regresión} logística no tiene el mejor rendimiento, al ser un clasificador, es interesante emplearlo, junto con regularización Lasso, para ver la evolución de los pesos de las variables para cada una de las clases, lo que aporta una mayor profundidad en el análisis.
El estudio se ha realizado sin tener en cuenta \textit{T1} y \textit{T2} y podemos ver que para las notas más bajas las variables relacionadas con la vida social y la familia son las más importantes, mientras que para las notas más altas, influyen variables más variadas.
Es interesante ver que para las notas más altas (Clase 20), muchas de las variables relacionadas con la vida social (\textit{AlcSem, AlcFin} y \textit{faltas}) se comportan de manera casi idéntica, lo que refleja un grado de relación grande, que ya hemos visto en el análisis anterior.

\begin{figure}[H]
  \centering
  \begin{minipage}[b]{0.3\textwidth}
      \centering
      \includegraphics[scale=0.25]{FI_Class1.png}
      \caption*{(a) Clase 1}
  \end{minipage}
  \hfill
  \begin{minipage}[b]{0.3\textwidth}
      \centering
      \includegraphics[scale=0.25]{FI_Class10.png}
      \caption*{(b) Clase 10}
  \end{minipage}
  \hfill
  \begin{minipage}[b]{0.3\textwidth}
      \centering
      \includegraphics[scale=0.25]{FI_Class20.png}
      \caption*{(c) Clase 20}
  \end{minipage}
  \caption{Feature Importance para tres clases}
\end{figure}


\section{Implementación y comparación de los dos modelos predictivos}

En esta sección se presentan los algoritmos con mayor rendimiento para ambos enfoques, junto con los criterios que justifican la elección de los modelos finales.

El desarrollo completo del análisis se encuentra en \textit{notebooks}. Las pruebas realizadas sobre los datos con los modelos iniciales se encuentran en \textit{Model1\_Testing} y \textit{Model2\_Testing}.
La selección de los parámetros de los modelos finales por validación cruzada se puede consultar en \textit{Final\_Model1\_Tuning} y \textit{Final\_Model2\_Tuning}. Por último, las métricas mostradas en el informe
se obtuvieron en \textit{Metrics\_Evaluator}.

\subsection{Modelo 1: Enfoque Predictivo con la Información Completa del Dataset}

Tras analizar múltiples modelos, se concluyó que los regresores superan consistentemente a los clasificadores, con una mejora media del 20\% en las métricas. Por ello, se descartó en una fase inicial continuar con clasificadores y se priorizó profundizar en modelos de regresión.

Los modelos de clasificación probados junto con su \textit{Accuracy} media fueron \textbf{Logistic Regression} (0.28), \textbf{Logistic Regression con Kernel-PCA} (0.57), \textbf{Support Vector Classifier (SVC)} (0.31), \textbf{Random Forest Classifier} (0.46) y  \textbf{Bagging Classifier} (0.45). 
Se programaron manualmente Logistic Regression, Random Forest Classifier y  Bagging Classifier.

La Tabla~\ref{table-M1} muestra los mejores modelos de regresión. Para garantizar la consistencia de las métricas, los valores presentados se corresponden con los promedios obtenidos tras entrenar 900 modelos por cada técnica, utilizando datos reordenados aleatoriamente en cada iteración.

Otros modelos de regresión entrenados, junto con sus valores \textit{$R^2$} medios fueron \textbf{Support Vector Regressor (SVR) con Kernel-PCA} (0.80), \textbf{Linear Regression con ISOMAP} (0.61) y \textbf{Linear Regression con Kernel-PCA} (0.80). Todos ellos fueron descartados para el proceso de validación cruzada ya que tienen una \textit{performance} inferior.
Los modelos de árboles individuales fueron descartados a favor de los \textit{Ensembles}.

\begin{table}[H]
  \caption{Modelos de regresión (Modelo 1)}
  \label{table-M1}
  \centering
  \begin{tabular}{lccccc}
    \toprule
    \multicolumn{6}{c}{Comparativa entre modelos} \\
    \cmidrule(r){1-6}
    Nombre & R² Score Test & MAE & MSE & R² Score Train & $\Delta$Score \\
    \midrule
    Linear Regression & 0.8301 $\pm$ 0.0335 & 0.9968 & 2.6484 & 0.853 & 0.0229\\
    LR con relaciones & 0.8338 $\pm$ 0.0327 & 1.0010 & 2.6682 & 0.854 & 0.0207 \\
    Ensemble de LR    & 0.8320 $\pm$ 0.0334 & 1.0011 & 2.6847 & 0.852 & 0.0200\\
    Stack             & 0.8497 $\pm$ 0.0334 & 0.9667 & 2.4228 & 0.938 & 0.0883\\
    Bagging           & 0.8531 $\pm$ 0.0366 & 0.9271 & 2.3367 & 0.979 & 0.1259\\
    Random Forest     & 0.8552 $\pm$ 0.0343 & 0.9119 & 2.2354 & 0.980 & 0.1248\\
    Boosting     & \textbf{0.8661 $\pm$ 0.0304} & \textbf{0.8945} & \textbf{2.1335} & 0.926 & 0.0597\\
    SVR               & 0.8331 $\pm$ 0.0371 & 0.9265 & 2.7643 & 0.838 & \textbf{0.0049}\\
    \bottomrule
  \end{tabular}
\end{table}

Podemos observar que \textbf{Gradient Boosting Regressor} tiene los mejores resultados en todas las métricas, y que además, al contrario que Bagging y Random Forest, tiene una $\Delta$Score muy baja, lo que indica poco \textit{overfitting}.
En este modelo ha sido muy importante la selección de parámetros, puesto que su \textit{score} era de los más variables (0.7-0.9).

\begin{figure}[ht]
  \centering
  \includegraphics[width=0.3\textwidth]{CV_Boosting_N.png}
  \caption{Proceso CV para Boosting para \textit{n\_estimators}.}
\end{figure}

\subsection{Modelo 2: Enfoque Predictivo sin las variables T1 y T2}

Al eliminar las variables de más peso, los modelos reducen a la mitad su poder predictivo, pero se mantiene la superioridad de los regresores frente a los clasificadores.

En esta ocasión, dados los resultados anteriores, los únicos clasificadores probados, junto con su \textit{Accuracy} media fueron \textbf{Logistic Regression} (0.237) y \textbf{Classification Stack}\footnote{Usando Logistic Regression, SVC y Random Forest Classification} (0.137).
En el Modelo 1, la \textit{performance} del regresor logístico mejoró al combinarlo con \textbf{Kernel-PCA}, pero con el 2º Modelo, se ha disminuido su \textit{Accuracy} a 0.0933. Esto se debe probablemente a que los primeros componentes principales capturaban en gran medida la varianza explicada 
por las variables T1 y T2. Al eliminarlas, la estructura de varianza se ve alterada significativamente, lo que reduce la capacidad del modelo.

Adicionalmente a los regresores mostrados en la tabla, se probaron a su vez \textbf{Linear Regression} (0.21), \textbf{Linear Regression con Clustering} (0.18), \textbf{Linear Regression con Kernel-PCA} (0.11) y mi clase personalizada de \textbf{Linear Regression con relaciones} (0.23). Dado que 
en el modelo anterior Gradient Boosting tuvo la mejor \textit{performance}, se investigaron métodos optimizados como \textbf{CatBoosting} (0.26), \textbf{XGBoost} (0.20) y \textbf{Light GBM} (0.23). Todos estos métodos de Boosting 
están optimizados para tratar con características como datasets grandes, muchas variables categóricas o gran dimensionalidad, características que coinciden con nuestro dataset, pero tras probarlos, fueron descartados al tener una \textit{performance} inferior a Gradient Boosting.

\begin{table}[H]
  \caption{Modelos de regresión (Modelo 2)}
  \label{table-M2}
  \centering
  \begin{tabular}{lccccc}
    \toprule
    \multicolumn{6}{c}{Comparativa entre modelos} \\
    \cmidrule(r){1-6}
    Nombre & R² Score Test & MAE & MSE & R² Score Train & $\Delta$Score \\
    \midrule
    Linear Regression  & 0.1999 $\pm$ 0.0625 & 2.5932 & 12.763 & 0.302 & 0.1021\\
    Stack              & 0.3153 $\pm$ 0.0699 & 2.4093 & 10.817 & 0.783 & 0.4677 \\
    Bagging            & 0.2998 $\pm$ 0.0806 & 2.3954 & 10.971 & 0.904 & 0.6042\\
    Random Forest      & 0.3056 $\pm$ 0.0768 & 2.3953 & 10.935 & 0.905 & 0.5994\\
    Boosting           & 0.3005 $\pm$ 0.0810 & 2.4709 & 11.025 & 0.681 & 0.3805\\
    SVR                & 0.2856 $\pm$ 0.0657 & 2.4193 & 11.423 & 0.730 & 0.4444\\
    \bottomrule
  \end{tabular}
\end{table}

\section{Conclusiones accionables}

El análisis de los datos nos indica que influyen tres grupos principales de variables a la hora de predecir el desempeño de los estudientes, las \textbf{notas previas}, la \textbf{vida social} del estudiante, y su \textbf{entorno familiar}.
Con esta información, los centros educativos pueden enfocar sus medidas en estas áreas. Sensibilizar al alumnado sobre la importancia de equilibrar vida social y estudio puede favorecer su rendimiento. Asimismo, promover políticas de 
conciliación familiar podría tener un impacto positivo en sus resultados académicos.


\section*{References}

{
\small


[1] Javier Béjar. \ \textit{Strategies and Algorithms for Clustering Large Datasets: A
Review.} \ Universidad Politécnica de Cataluña.
\url{https://upcommons.upc.edu/bitstream/handle/2117/23415/R13-11.pdf}


[2] Alfonso Cervantes Barragan. \ (2024). \ \textit{Interpreting and Validating Clustering Results with K-Means}. 
\ Medium. 
\url{https://medium.com/@a.cervantes2012/interpreting-and-validating-clustering-results-with-k-means-e98227183a4d}


[3] Connie Zhou.\ (2023). \ \textit{Unraveling Data Patterns with Isomap: A Guide to Dimensionality Reduction — Part 4}.
\url{https://medium.com/@conniezhou678/unraveling-data-patterns-with-isomap-a-guide-to-dimensionality-reduction-part-4-1d774eee69a5}

[4] (2025). \ \textit{Gradient Boosting in ML}. \ Geeks for Geeks.
\url{https://www.geeksforgeeks.org/ml-gradient-boosting/}

}






The only supported style file for NeurIPS 2023 is \verb+neurips_2023.sty+,
rewritten for \LaTeXe{}.  \textbf{Previous style files for \LaTeX{} 2.09,
  Microsoft Word, and RTF are no longer supported!}


The \LaTeX{} style file contains three optional arguments: \verb+final+, which
creates a camera-ready copy, \verb+preprint+, which creates a preprint for
submission to, e.g., arXiv, and \verb+nonatbib+, which will not load the
\verb+natbib+ package for you in case of package clash.


\paragraph{Preprint option}
If you wish to post a preprint of your work online, e.g., on arXiv, using the
NeurIPS style, please use the \verb+preprint+ option. This will create a
nonanonymized version of your work with the text ``Preprint. Work in progress.''
in the footer. This version may be distributed as you see fit, as long as you do not say which conference it was submitted to. Please \textbf{do
  not} use the \verb+final+ option, which should \textbf{only} be used for
papers accepted to NeurIPS. 


At submission time, please omit the \verb+final+ and \verb+preprint+
options. This will anonymize your submission and add line numbers to aid
review. Please do \emph{not} refer to these line numbers in your paper as they
will be removed during generation of camera-ready copies.


The file \verb+neurips_2023.tex+ may be used as a ``shell'' for writing your
paper. All you have to do is replace the author, title, abstract, and text of
the paper with your own.



\section{General formatting instructions}
\label{gen_inst}


The text must be confined within a rectangle 5.5~inches (33~picas) wide and
9~inches (54~picas) long. The left margin is 1.5~inch (9~picas).  Use 10~point
type with a vertical spacing (leading) of 11~points.  Times New Roman is the
preferred typeface throughout, and will be selected for you by default.
Paragraphs are separated by \nicefrac{1}{2}~line space (5.5 points), with no
indentation.


The paper title should be 17~point, initial caps/lower case, bold, centered
between two horizontal rules. The top rule should be 4~points thick and the
bottom rule should be 1~point thick. Allow \nicefrac{1}{4}~inch space above and
below the title to rules. All pages should start at 1~inch (6~picas) from the
top of the page.


For the final version, authors' names are set in boldface, and each name is
centered above the corresponding address. The lead author's name is to be listed
first (left-most), and the co-authors' names (if different address) are set to
follow. If there is only one co-author, list both author and co-author side by
side.


Please pay special attention to the instructions in Section 
regarding figures, tables, acknowledgments, and references.


\section{Headings: first level}
\label{headings}


All headings should be lower case (except for first word and proper nouns),
flush left, and bold.


First-level headings should be in 12-point type.


\subsection{Headings: second level}


Second-level headings should be in 10-point type.


\subsubsection{Headings: third level}


Third-level headings should be in 10-point type.


\paragraph{Paragraphs}


There is also a \verb+\paragraph+ command available, which sets the heading in
bold, flush left, and inline with the text, with the heading followed by 1\,em
of space.


\section{Citations, figures, tables, references}
\label{others}


These instructions apply to everyone.


\subsection{Citations within the text}


The \verb+natbib+ package will be loaded for you by default.  Citations may be
author/year or numeric, as long as you maintain internal consistency.  As to the
format of the references themselves, any style is acceptable as long as it is
used consistently.


The documentation for \verb+natbib+ may be found at
\begin{center}
  \url{http://mirrors.ctan.org/macros/latex/contrib/natbib/natnotes.pdf}
\end{center}
Of note is the command \verb+\citet+, which produces citations appropriate for
use in inline text.  For example,
\begin{verbatim}
   \citet{hasselmo} investigated\dots
\end{verbatim}
produces
\begin{quote}
  Hasselmo, et al.\ (1995) investigated\dots
\end{quote}


If you wish to load the \verb+natbib+ package with options, you may add the
following before loading the \verb+neurips_2023+ package:
\begin{verbatim}
   \PassOptionsToPackage{options}{natbib}
\end{verbatim}


If \verb+natbib+ clashes with another package you load, you can add the optional
argument \verb+nonatbib+ when loading the style file:
\begin{verbatim}
   \usepackage[nonatbib]{neurips_2023}
\end{verbatim}


As submission is double blind, refer to your own published work in the third
person. That is, use ``In the previous work of Jones et al.\ [4],'' not ``In our
previous work [4].'' If you cite your other papers that are not widely available
(e.g., a journal paper under review), use anonymous author names in the
citation, e.g., an author of the form ``A.\ Anonymous'' and include a copy of the anonymized paper in the supplementary material.


\subsection{Footnotes}


Footnotes should be used sparingly.  If you do require a footnote, indicate
footnotes with a number\footnote{Sample of the first footnote.} in the
text. Place the footnotes at the bottom of the page on which they appear.
Precede the footnote with a horizontal rule of 2~inches (12~picas).


Note that footnotes are properly typeset \emph{after} punctuation
marks.\footnote{As in this example.}


\subsection{Figures}


\begin{figure}
  \centering
  \fbox{\rule[-.5cm]{0cm}{4cm} \rule[-.5cm]{4cm}{0cm}}
  \caption{Sample figure caption.}
\end{figure}


All artwork must be neat, clean, and legible. Lines should be dark enough for
purposes of reproduction. The figure number and caption always appear after the
figure. Place one line space before the figure caption and one line space after
the figure. The figure caption should be lower case (except for first word and
proper nouns); figures are numbered consecutively.


You may use color figures.  However, it is best for the figure captions and the
paper body to be legible if the paper is printed in either black/white or in
color.


\subsection{Tables}


All tables must be centered, neat, clean and legible.  The table number and
%title always appear before the table.  See Table~\ref{sample-table}.


Place one line space before the table title, one line space after the
table title, and one line space after the table. The table title must
be lower case (except for first word and proper nouns); tables are
numbered consecutively.


Note that publication-quality tables \emph{do not contain vertical rules.} We
strongly suggest the use of the \verb+booktabs+ package, which allows for
typesetting high-quality, professional tables:
\begin{center}
  \url{https://www.ctan.org/pkg/booktabs}
\end{center}
%This package was used to typeset Table~\ref{sample-table}.



\subsection{Math}
%Note that display math in bare TeX commands will not create correct line numbers for submission. Please use LaTeX (or AMSTeX) commands for unnumbered display math. (You really shouldn't be using \$\$ anyway; see \url{https://tex.stackexchange.com/questions/503/why-is-preferable-to} and \url{https://tex.stackexchange.com/questions/40492/what-are-the-differences-between-align-equation-and-displaymath} for more information.)

\subsection{Final instructions}

Do not change any aspects of the formatting parameters in the style files.  In
particular, do not modify the width or length of the rectangle the text should
fit into, and do not change font sizes (except perhaps in the
\textbf{References} section; see below). Please note that pages should be
numbered.


\section{Preparing PDF files}


Please prepare submission files with paper size ``US Letter,'' and not, for
example, ``A4.''


Fonts were the main cause of problems in the past years. Your PDF file must only
contain Type 1 or Embedded TrueType fonts. Here are a few instructions to
achieve this.


\begin{itemize}


\item You should directly generate PDF files using \verb+pdflatex+.


\item You can check which fonts a PDF files uses.  In Acrobat Reader, select the
  menu Files$>$Document Properties$>$Fonts and select Show All Fonts. You can
  also use the program \verb+pdffonts+ which comes with \verb+xpdf+ and is
  available out-of-the-box on most Linux machines.


\item \verb+xfig+ "patterned" shapes are implemented with bitmap fonts.  Use
  "solid" shapes instead.


\item The \verb+\bbold+ package almost always uses bitmap fonts.  You should use
  the equivalent AMS Fonts:
\begin{verbatim}
   \usepackage{amsfonts}
\end{verbatim}
followed by, e.g., \verb+\mathbb{R}+, \verb+\mathbb{N}+, or \verb+\mathbb{C}+
for $\mathbb{R}$, $\mathbb{N}$ or $\mathbb{C}$.  You can also use the following
workaround for reals, natural and complex:
\begin{verbatim}
   \newcommand{\RR}{I\!\!R} %real numbers
   \newcommand{\Nat}{I\!\!N} %natural numbers
   \newcommand{\CC}{I\!\!\!\!C} %complex numbers
\end{verbatim}
Note that \verb+amsfonts+ is automatically loaded by the \verb+amssymb+ package.


\end{itemize}


If your file contains type 3 fonts or non embedded TrueType fonts, we will ask
you to fix it.


\subsection{Margins in \LaTeX{}}


Most of the margin problems come from figures positioned by hand using
\verb+\special+ or other commands. We suggest using the command
\verb+\includegraphics+ from the \verb+graphicx+ package. Always specify the
figure width as a multiple of the line width as in the example below:
\begin{verbatim}
   \usepackage[pdftex]{graphicx} ...
   \includegraphics[width=0.8\linewidth]{myfile.pdf}
\end{verbatim}
See Section 4.4 in the graphics bundle documentation
(\url{http://mirrors.ctan.org/macros/latex/required/graphics/grfguide.pdf})


A number of width problems arise when \LaTeX{} cannot properly hyphenate a
line. Please give LaTeX hyphenation hints using the \verb+\-+ command when
necessary.


\begin{ack}
Use unnumbered first level headings for the acknowledgments. All acknowledgments
go at the end of the paper before the list of references. Moreover, you are required to declare
funding (financial activities supporting the submitted work) and competing interests (related financial activities outside the submitted work).
More information about this disclosure can be found at: \url{https://neurips.cc/Conferences/2023/PaperInformation/FundingDisclosure}.


Do {\bf not} include this section in the anonymized submission, only in the final paper. You can use the \texttt{ack} environment provided in the style file to autmoatically hide this section in the anonymized submission.
\end{ack}



\section{Supplementary Material}

Authors may wish to optionally include extra information (complete proofs, additional experiments and plots) in the appendix. All such materials should be part of the supplemental material (submitted separately) and should NOT be included in the main submission.




%%%%%%%%%%%%%%%%%%%%%%%%%%%%%%%%%%%%%%%%%%%%%%%%%%%%%%%%%%%%


\end{document}